\documentclass{src/thesis}  % uncomment for writing
% \documentclass[print]{src/thesis} % uncomment for printing version

% ------------------- PACKAGES ------------------- %
\usepackage{amsmath}        %Used for math
\usepackage{amsthm}         %Used for intermezzos
\usepackage{lipsum}         %Used for dummy text
\usepackage{longtable}      %Used to put nomenclature on multiple pages
\usepackage{multicol}       %For multiple columns in table
\usepackage{multirow}       %For multiple rows in table
\usepackage{tcolorbox}      %For statements and contributions
\usepackage{thmtools}       %For declaring theorems
\usepackage{todonotes}      %For to do notes
%\usepackage{tikz}          %The tikz package is added in mrthesis.cls for producing thumbindex
\usepackage{url}            %To provide urls in the document
\usepackage[hidelinks]{hyperref} %For hyperlinks in the document
\usepackage{bookmark}       %Works with hyperref to setup styles and colors
\usepackage{enumitem}       %To shift items in itemize left
\usepackage{wrapfig}        %To wrap text around a figure






% ------------------- PREAMBLE ------------------- %
% Counters
\newcounter{ContNum}        % Counter for the contributions
\renewcommand{\theContNum}{\Roman{ContNum}}

%Other commands
\newcommand{\itemheader}[1]{~\\ \noindent\textbf{#1.}\ \ }%
\newcommand{\itemheaderNewpage}[1]{\newpage \noindent\textbf{#1.}\ \ }%
\newcommand{\contribution}[2]{\refstepcounter{ContNum}#2 \vspace*{2.1mm}\begin{tcolorbox}[colback=black!2!white,colframe=black!20!white] \textbf{Contribution \Roman{ContNum}.} {\em #1} \end{tcolorbox}\vspace*{2.1mm}}
\newcommand{\objective}[1]{\begin{tcolorbox}[colback=black!2!white,colframe=black!20!white] {\em #1} \end{tcolorbox}}
\newcommand{\terminology}[2]{\begin{tcolorbox}[colback=black!2!white,colframe=black!20!white] {\textbf{Terminology: } \textbf{#1} \em #2} \end{tcolorbox}}
\newcommand{\cover}[1]{\ifprint{}\else\includepdf[pages=-]{#1}\cleardoublepage\fi}


% To create a blank footnote:
\newcommand\blfootnote[1]{%
  \let\thefootnote\relax\footnotetext{#1}%
  \let\thefootnote\svthefootnote%
}
\newcounter{numfootnote}
\newcommand\numfootnote[1]{%
    \stepcounter{numfootnote}%
    \newcommand{\thefootnote}{\thenumfootnote}%
    \footnote{#1}
}


% Referring to the Modifications chapter:
\newcommand{\disclaimer}{\\[\baselineskip]  A detailed list of the differences between this chapter and the article on which it is based is provided in the \hyperref[chap: Modifications]{\textit{Modifications}} chapter of this thesis.}
%!TEX root = ../thesis.tex
\usepackage{tikz}
\usepackage{pgfplots}
% and optionally (as of Pgfplots 1.3):
\pgfplotsset{compat=newest}
\pgfplotsset{plot coordinates/math parser=false}
\newlength\figureheight
\newlength\figurewidth

\usepackage[utf8]{inputenc}
\usepackage{pgfgantt}
\usepackage{pdflscape}
\pgfplotsset{compat=newest}
\pgfplotsset{plot coordinates/math parser=false}

% %% the following commands are needed for some matlab2tikz features
\usetikzlibrary{plotmarks}
\usetikzlibrary{arrows.meta}
\usepgfplotslibrary{patchplots}

\newcommand{\SF}{1}                 % Scaling factor
\newcommand{\TS}{\normalsize}       % Text size
\newcommand{\lw}{0.7pt}             % Line width
\newcommand{\TSTick}{\small}        % Text size axis labels
\newcommand{\axislabels}[2]{\foreach \x/\y/\s in {#1} {\node[#2,inner sep=1mm] at (\x,\y) {\TSTick $\s$};}}
\newcommand{\wheel}[3]{ \draw[line width=\lw] (#1,#2) circle (#3);
                        \fill[bottom color=MRblue!60!black!80,top color=MRblue!10] (#1,#2) circle (#3-0.5*\lw);
                        \fill[color=MRblue!30] (#1,#2) circle (#3-\lw);
                        \fill[top color=MRblue!60!black!80,bottom color=MRblue!10] (#1,#2) circle (0.7*#3);
                        \fill[color=MRblue!30] (#1,#2) circle (0.7*#3-\lw);
                        \fill[color=black] (#1,#2) circle (0.1*#3);}

\newcommand{\CoM}[3]{\filldraw[inner color=white,outer color=black!7!white,draw=black,line width=\lw] (#1,#2) circle (#3+0.5*\lw);
                     \begin{scope}[xshift=#1,yshift=#2]
                        \clip(-#3,0) -- (0,0) -- (0,#3) -- (#3,#3) -- (#3,0) -- (0,0) -- (0,-#3) -- (-#3,-#3) -- cycle;
                        \fill[inner color=black!50!white,outer color=black] (0,0) circle (#3);
                     \end{scope}}
\usetikzlibrary{arrows}
\usetikzlibrary{patterns}
\usetikzlibrary{decorations.markings}
\usetikzlibrary{shadings}
\usetikzlibrary{shapes}


\declaretheorem[name={Remark},style=plain,numberwithin=chapter]{rmk}
\declaretheorem[name={Definition},style=definition,numberwithin=chapter]{dfn}
\declaretheorem[name={Theorem},qed=$\blacktriangle$,style=plain,numberwithin=chapter]{thm}
\declaretheorem[name={Lemma},qed=$\blacktriangle$,style=plain,numberwithin=chapter]{lem}
\declaretheorem[name={Assumption},style=definition,numberwithin=chapter]{asm}
\declaretheorem[name={Property},style=plain,numberwithin=chapter]{prop}
\declaretheorem[name={Example},style=plain,numberwithin=chapter]{example}
\declaretheorem[name={Corollary},qed=$\blacktriangle$,style=plain,numberwithin=chapter]{cor}
\declaretheorem[name={Intermezzo},style=definition,numberwithin=chapter]{intermez}

\graphicspath{{img/}{../../img/}{../../../img/}}


% ------------------- Document colors -------------------%
% DEEP BLUE
\definecolor{thumbcolor}{RGB}{250,180,140}
\definecolor{deepcolor}{RGB}{0,0,0}
\definecolor{titlecolor}{RGB}{238,112,35}


% ------------------- Document Details -------------------%
%Author details
\author{Author Name}
\newcommand{\placeofbirth}{Birthplace}

%Defense details
\renewcommand{\year}{2023}
\newcommand{\defensedate}{maandag 30 december 2022}
\newcommand{\defensetime}{16:00}
\newcommand{\rector}{prof. dr. Rector}

%Thesis details
\newcommand{\maintitle}{Awesome thesis title}
\newcommand{\subtitle}{with an awesome subtitle}
\newcommand{\isbn}{123-45-678-9012-3}     % 123boldmath-45-678-9012-3
\newcommand{\printer}{Name of printer || www.printerswebsite.nl}
\newcommand{\designer}{Name of cover designer}
\newcommand{\project}{Here you can put the acknowledgment in case this thesis is partially 
supported by the Research Project XXX through the European Union H2020 
program under GA XXXXXX.}

%Details about the committee (note that there should be NO space between titles (prof.dr.)
\newcommand{\chair}{prof.dr. Name Surname}
\newcommand{\promotor}{prof.dr. Name Surname}
\newcommand{\copromotor}{prof.dr. Name Surname}
\newcommand{\firstmember}{prof.dr. Name Surname}
\newcommand{\secondmember}{prof.dr. Name Surname}
\newcommand{\thirdmember}{prof.dr. Name Surname}
\newcommand{\fourthmember}{prof.dr. Name Surname}
% other relevant meta data about the thesis can be found in preface.tex

\makeatletter % generates all the \author stuff

\begin{document} 
\pagenumbering{roman}
\thispagestyle{empty}

% ------------------- Front Cover -------------------%
% \cover{thesis_front.pdf}    % Adds your thesis' front cover if the option 'print' is NOT used. Printing companies require the thesis cover to be supplied separately and in a different format. Definition of \cover{} is given in commands.tex.

% ------------------- First Sections -------------------%
% ---------------- First page: title and name ---------------- %
\thispagestyle{empty}
\vspace*{30mm}\noindent
\begin{center}
{\LARGE\sf\maintitle}\\[4.5cm] %\\[7mm]
{\Large\sf \@author}
\end{center}

\newpage
\thispagestyle{empty}






% --------------- Second page: acknowledgments --------------- %
%TUE logo 
\vspace*{\fill}
\noindent\includegraphics[width=5cm]{img/TUE-logo.pdf}\\
{\small The work described in this thesis was carried out at the Eindhoven University of
Technology.}\\[8mm]

% Project Logo %
\noindent\includegraphics[width=5cm]{img/ProjectLogo.pdf}\\[2mm]
\noindent\bgroup\small\project
\\[8mm]

%ISBN%
\noindent\bgroup\small
A catalogue record is available from the Eindhoven University of Technology Library.\\
ISBN: \isbn\\[4mm]

%Other%
Typeset by the author using the pdf \LaTeX \ documentation system.\\
Cover design: \designer \\
Reproduction: \printer\\[8mm]
\copyright\year \ by \@author. All rights reserved.}
\egroup

\newpage
\thispagestyle{empty}





% ------------------- Third page: Title page ------------------- %
\vspace*{30mm}
\begin{center}
{\LARGE\sf\maintitle}\\[30mm] %\\[7mm]
{\large\textsc{Proefschrift}}\\[8mm]
ter verkrijging van de graad van doctor aan de\\
Technische Universiteit Eindhoven, op gezag van de\\
rector magnificus \rector, voor een\\
commissie aangewezen door het College voor\\
Promoties, in het openbaar te verdedigen\\
op \defensedate\ om \defensetime\ uur\\[8mm]
door\\[8mm]
\@author\\[8mm]
geboren te \placeofbirth
\end{center}
\vfill

\newpage
\thispagestyle{empty}

\noindent
Dit proefschrift is goedgekeurd door de promotoren en de samenstelling van de promotiecommissie is als volgt:\\[7mm]

\noindent
\begin{tabular}{@{}l p{9.8cm}}
voorzitter:                 &   \chair        \\                \\
promotor:                   &   \promotor     \\
co-promotor:                &   \copromotor   \\
leden:                      &   \firstmember  \\
                            &   \secondmember \\
                            &   \thirdmember  \\
                            &   \fourthmember \\
\end{tabular}

\vfill
\noindent
Het onderzoek dat in dit proefschrift wordt beschreven is uitgevoerd in overeenstemming met de TU/e Gedragscode Wetenschapsbeoefening.

\include{02_abstract}
\include{03_samenvatting}
\include{04_societal_summary}

% ------------------- Nomenclature -------------------%
\cleardoublepage
\pdfbookmark{\contentsname}{Contents}
\tableofcontents

\newpage
\include{05_nomenclature}

%%%% MAIN MATTERS *************************************************************
% This state variable is used for creating the thumb index by indicating that
% the following chapter is numbered (i.e. not \chapter*{})
\placethumbtrue %Place thumbs from this point onwards
\cleardoublepage
\thispagestyle{empty}
\setcounter{page}{1}
\pagenumbering{arabic}
\part{Background}\label{part: intro}
\include{06_CH1_Introduction}

% ------------------- FIRST PART -------------------%
\cleardoublepage
\part{Second chapter}\label{part: design}
\chapter{Title of the chapter}\label{chap: chapter 1}

\blfootnote{This chapter is based on:\\ write down the citation of the paper you are referring to in full.\disclaimer}


\chapterabstract{\lipsum[1]}

\section{Introduction} \label{sec: chap2 intro}
Showing footnotes still work even though we use blank footnotes\footnote{You see?}
\lipsum[5-7]

\section{Section header}\label{sec: chap2 section header}

Here's an equation:
\begin{equation}\label{eq: chap2 vector field} 
x = 1
\end{equation}

with $x$ the input, $t_0$ the initial time, and $t_f$ the final time. 
And below another beautiful image.



\section{Conclusions and discussion}
\label{sec: chap2 conclusion}

\lipsum[10-13]




% ------------------- SECOND PART -------------------%
\cleardoublepage
\part{Third part}\label{part: something}
\include{06_CH3_ThirdChapter}

% ------------------- THIRD PART -------------------%
\part{Closing}\label{part: closing}
\include{06_CH4_conclusions}

% -------------------- APPENDIX --------------------%
\part{Appendices}\label{part: Appendices}
\appendix
\cleardoublepage

\definecolor{thumbcolor}{gray}{0.4}     % Change the color of the thumb index tags to gray
\include{07_appendix}

% ------------------- BACK PART -------------------%
\placethumbfalse   % Stop putting thumbs from this point onwards
\bookmarksetup{startatroot}
\addtocontents{toc}{\bigskip}%

% -------------- BIBLIOGRAPHY ------------ %
{\fontsize{9pt}{10pt}\selectfont
\bibliographystyle{IEEEtran}
\cleardoublepage
\phantomsection
\addcontentsline{toc}{chapter}{\bibname}
\bibliography{src/bibliography}
% \nobibliography{src/bibliography}

}

% -------------- MODIFICATIONS ------------- %
\chapter*{Modifications}\label{chap: Modifications}
\addcontentsline{toc}{chapter}{Modifications}
\markboth{Modifications}{Modifications}

The chapters of your thesis are likely based on publications. In this chapter you can put the modifications with respect to those publications in forming the thesis content.

\section*{Overall modifications}
\begin{itemize}
    \item Phrases like \textit{this work}, \textit{this paper}, and \textit{this manuscript} are replaced by \textit{this chapter}, or \textit{this thesis}
\end{itemize}

\section*{Modifications Chapter 2}
\begin{itemize}
    \item A modification
    \item Another modification
\end{itemize}


% ------------- ACKNOWLEDGEMENT ------------ %
\normalsize
\include{09_acknowledgement}

% --------------- PUBLICATIONS ------------- %
\newpage
%!TEX root = ../thesis.tex
\chapter*{List of publications}
\addcontentsline{toc}{chapter}{List of publications}
\markboth{List of publications}{List of publications}
\newcommand{\ipj}{(\textit{in preparation for journal submission})}
\newcommand{\cur}{(\textit{under review})}
\newcommand{\sbm}{(\textit{submitted})}
\newcommand{\acp}{(\textit{accepted})}
\newcommand{\inp}{(\textit{in press})}


%You think this can be done way easier with something like "bibentry" via the "bibentry" package.
%However, this also puts all references in the bibliography, which does not make sense if you don't 
%cite them... so only work around possible at this moment, is doing it manually :)

% -------------------------------- Journal papers -------------------------------- %
\begin{itemize}[leftmargin=4mm]
  \item B. Caasenbrood, A. Pogromsky and H. Nijmeijer, “\textit{Generative Design of Soft Robotic Actuators -- a Gradient-based Approach}”, Frontiers in Robotics and AI, 2022. \ipj;
\item B. Caasenbrood, A. Pogromsky and H. Nijmeijer, “\textit{Reduced-order Cosserat Models for Soft Robotic
 Systems using FEM-driven Shape Reconstruction}”, Robotics and Automation Letters, 2022. \ipj;
\item A. Amiri, B. Caasenbrood, D. Liu, N. van de Wouw, and I. Lopez Arteaga, "\textit{An Electric Circuit Model for the Nonlinear Dynamics of Electro-active Liquid Crystal Coatings}", Applied Physics Letters, 2022. \sbm;
\item  B. Caasenbrood, A. Pogromsky and H. Nijmeijer, “\textit{Energy-shaping Controllers for Soft Robot Manipulators through Port-Hamiltonian Cosserat Models}”, SN Computer Science Springer, 2022. \acp;
\item B. Caasenbrood, A. Pogromsky and H. Nijmeijer, "\textit{Control-oriented Models for Hyper-elastic Soft Robots through Differential Geometry of Curves}”, Soft Robotics, 2022. \inp
\end{itemize}

% ------------------------------ Conference papers ------------------------------ %
\section*{Peer-reviewed articles in conference proceedings}
\begin{itemize}[leftmargin=4mm]
\item B. Caasenbrood, F.E. van Beek, H. Khanh Chu, and I.A. Kuling, “\textit{A Desktop-sized Platform for Real-time Control Applications of Pneumatic Soft Robots},” IEEE International Conference on Soft Robotics, RoboSoft 2022, pp 217-223.
\item A. Amoozandeh Nobaveh, and B. Caasenbrood, "\textit{Design Feasibility of an Energy-efficient Wrist Exoskeleton
using Compliant Beams and Soft Actuators}", Proceedings of the 18th International  Consortium for Rehabilitation Robotics, 2022 (accepted).
\item B. Caasenbrood, A. Pogromsky and H. Nijmeijer, "\textit{Energy-based control for Soft Robots using Cosserat-beam models}”, Proceedings of the 18th International Conference on Informatics in Control, Automation and Robotics, 2021, pp. 311–319.
\item B. Caasenbrood, A. Pogromsky and H. Nijmeijer, "\textit{A Computational Design Framework for Pressure-driven Soft Robots through Nonlinear Topology Optimization}," 2020 3rd IEEE International Conference on Soft Robotics, 2020, pp. 633-638.
\item B. Caasenbrood, A. Pogromsky and H. Nijmeijer, “\textit{Dynamic modeling of hyper-elastic soft robots using spatial curves},” IFAC World Congress, IFAC-PapersOnLine, 2020, pp. 9238-9243.
\end{itemize}


% ---------------- Invited talks and Non Peer-reviewed Abstracts ---------------- %
\section*{Invited Talks and Non Peer-reviewed Abstracts}
\begin{itemize}[leftmargin=4mm]
\item B. Caasenbrood, “\texttt{SOROTOKI}\textit{: an Open-source Toolkit for Soft Robotics written in MATLAB},”  IEEE International Conference on Soft Robotics, RoboSoft 2022 (abstract). \texttt{Best Poster Award}
\item B. Caasenbrood, C. Della Santina, and A. Pogromsky, “\textit{Workshop on Model-based Control of Soft Robots},” European Control Conference (ECC), 2021. (main organizer).
\item B. Caasenbrood, talk on  “\textit{Towards Desing and Control of Soft Robotics},” 4TU Symposium on Soft Robotics, 2020. (invited speaker).
\item B. Caasenbrood, talk on  “\textit{3D-printed Soft Robotics},” Symposium on Robotic Technologies, 2019. (invited speaker).
\item B. Caasenbrood, A. Pogromsky and H. Nijmeijer, talk on  “\textit{Forward Dynamics of Hyper-elastic Soft Robotics},” 39th Benelux Meeting on Systems and Control, 2019. (abstract).
\item B. Caasenbrood, A. Pogromsky and H. Nijmeijer, talk on  “\textit{Dynamical modeling and control of continuum soft robots},” 37th Benelux Meeting on Systems and Control, 2018. (abstract).
\end{itemize}

% -------------------- CV ------------------ %
%*********************************************************************************%
% \chapter*{Curriculum Vitae}
% \addcontentsline{toc}{chapter}{Curriculum Vitae}
% \markboth{Curriculum Vitae}{Curriculum Vitae}

\chapter*{About the author}
\addcontentsline{toc}{chapter}{About the author}
\markboth{About the author}{About the author}


\begin{wrapfigure}{l}{0.25\textwidth}
    \centering
    \includegraphics[width=0.25\textwidth]{img/authorpicture.jpg}
\end{wrapfigure}

\lipsum[5-7]


\thispagestyle{empty}

\ifprint{}
\else
\newpage {~}
\thispagestyle{empty}
\fi

%%%% BACK **********************************************************************
% \cover{thesis_back.pdf}    % Adds your thesis' back cover if the option 'print' is NOT used. Printing companies require the thesis cover to be supplied separately and in a different format. Definition of \cover{} is given in commands.tex.

\end{document}