% Counters
\newcounter{ContNum}        % Counter for the contributions
\renewcommand{\theContNum}{\Roman{ContNum}}

%Other commands
\newcommand{\itemheader}[1]{~\\ \noindent\textbf{#1.}\ \ }%
\newcommand{\itemheaderNewpage}[1]{\newpage \noindent\textbf{#1.}\ \ }%
\newcommand{\contribution}[2]{\refstepcounter{ContNum}#2 \vspace*{2.1mm}\begin{tcolorbox}[colback=black!2!white,colframe=black!20!white] \textbf{Contribution \Roman{ContNum}.} {\em #1} \end{tcolorbox}\vspace*{2.1mm}}
\newcommand{\objective}[1]{\begin{tcolorbox}[colback=black!2!white,colframe=black!20!white] {\em #1} \end{tcolorbox}}
\newcommand{\terminology}[2]{\begin{tcolorbox}[colback=black!2!white,colframe=black!20!white] {\textbf{Terminology: } \textbf{#1} \em #2} \end{tcolorbox}}
\newcommand{\cover}[1]{\ifprint{}\else\includepdf[pages=-]{#1}\cleardoublepage\fi}


% To create a blank footnote:
\newcommand\blfootnote[1]{%
  \let\thefootnote\relax\footnotetext{#1}%
  \let\thefootnote\svthefootnote%
}
\newcounter{numfootnote}
\newcommand\numfootnote[1]{%
    \stepcounter{numfootnote}%
    \newcommand{\thefootnote}{\thenumfootnote}%
    \footnote{#1}
}


% Referring to the Modifications chapter:
\newcommand{\disclaimer}{\\[\baselineskip]  A detailed list of the differences between this chapter and the article on which it is based is provided in the \hyperref[chap: Modifications]{\textit{Modifications}} chapter of this thesis.}