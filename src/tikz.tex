%!TEX root = ../thesis.tex
\usepackage{tikz}
\usepackage{pgfplots}
% and optionally (as of Pgfplots 1.3):
\pgfplotsset{compat=newest}
\pgfplotsset{plot coordinates/math parser=false}
\newlength\figureheight
\newlength\figurewidth

\usepackage[utf8]{inputenc}
\usepackage{pgfgantt}
\usepackage{pdflscape}
\pgfplotsset{compat=newest}
\pgfplotsset{plot coordinates/math parser=false}

% %% the following commands are needed for some matlab2tikz features
\usetikzlibrary{plotmarks}
\usetikzlibrary{arrows.meta}
\usepgfplotslibrary{patchplots}

\newcommand{\SF}{1}                 % Scaling factor
\newcommand{\TS}{\normalsize}       % Text size
\newcommand{\lw}{0.7pt}             % Line width
\newcommand{\TSTick}{\small}        % Text size axis labels
\newcommand{\axislabels}[2]{\foreach \x/\y/\s in {#1} {\node[#2,inner sep=1mm] at (\x,\y) {\TSTick $\s$};}}
\newcommand{\wheel}[3]{ \draw[line width=\lw] (#1,#2) circle (#3);
                        \fill[bottom color=MRblue!60!black!80,top color=MRblue!10] (#1,#2) circle (#3-0.5*\lw);
                        \fill[color=MRblue!30] (#1,#2) circle (#3-\lw);
                        \fill[top color=MRblue!60!black!80,bottom color=MRblue!10] (#1,#2) circle (0.7*#3);
                        \fill[color=MRblue!30] (#1,#2) circle (0.7*#3-\lw);
                        \fill[color=black] (#1,#2) circle (0.1*#3);}

\newcommand{\CoM}[3]{\filldraw[inner color=white,outer color=black!7!white,draw=black,line width=\lw] (#1,#2) circle (#3+0.5*\lw);
                     \begin{scope}[xshift=#1,yshift=#2]
                        \clip(-#3,0) -- (0,0) -- (0,#3) -- (#3,#3) -- (#3,0) -- (0,0) -- (0,-#3) -- (-#3,-#3) -- cycle;
                        \fill[inner color=black!50!white,outer color=black] (0,0) circle (#3);
                     \end{scope}}
\usetikzlibrary{arrows}
\usetikzlibrary{patterns}
\usetikzlibrary{decorations.markings}
\usetikzlibrary{shadings}
\usetikzlibrary{shapes}
